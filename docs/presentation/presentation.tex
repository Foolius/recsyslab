%\documentclass[mathserif,handout]{beamer}
\documentclass[mathserif,svgnames]{beamer}
%
\usepackage{amsmath}
\usepackage{amssymb}
\usepackage{graphicx}
\usepackage{epsfig}
\usepackage{color}
%\usepackage{biblatex}
\hypersetup{breaklinks=false}

%Define the l3s color
%RGB=50,78,125
\xdefinecolor{l3scolor}{rgb}{0.2,0.31,0.49}

%color structure for template
\usecolortheme[named=l3scolor]{structure}

%color definition for elements
\setbeamercolor*{Title bar}{fg=white,bg=l3scolor}
\setbeamercolor*{Location bar}{fg=l3scolor,bg=black}
%\setbeamercolor*{frametitle}{parent=Title bar}
\setbeamercolor*{normal text}{bg=white,fg=l3scolor}
\setbeamercolor*{section in head/foot}{bg=black,fg=white}
\setbeamercolor*{date in head/foot}{bg=black,fg=white}
\setbeamercolor*{alerted text}{fg=black}
\setbeamercolor*{block body alerted}{bg=l3scolor!14}
\setbeamercolor*{block title alerted}{bg=l3scolor!36} 

%Blocks are rounded with a shadow
\setbeamertemplate{blocks}[rounded][shadow=true]

%watermark logo
\logo{\includegraphics[width=11cm]{l3s_watermark.jpg}}


\setbeamerfont{section in head/foot}{size=\tiny,series=\normalfont}
\setbeamerfont{frametitle}{size=\Large}
\defbeamertemplate*{footline}{infolines theme}

\usepackage{rotating}
%

\setbeamertemplate{headline}{
\includegraphics[width=\paperwidth]{header}
}
%
\setbeamertemplate{footline}{
	\hspace{10mm}\colouredline{l3scolor!14}{
	\color{gray}
	\hspace{165pt}  \textbf{Julius Kolbe} $\,\mid\,$ \insertshorttitle $\,\mid\,$ \insertframenumber/\inserttotalframenumber \hfill
	}
}

\usenavigationsymbolstemplate{}
\title[Library of Recommender System Algorithms]
{Multi-purpose Library of Recommender System Algorithms for the Item Prediction Task}
\subtitle{Presentation of my Bachelor Thesis}
\author{Julius Kolbe}
%\institute{Fakult\"at f\"ur Elektrotechnik und Informatik\\Institut f\"ur Verteilte Systeme}
\date{\today}
\usepackage{pdfpages}
\usepackage{listings}
\lstdefinestyle{pseudocode}{%language=Python,
    keywordstyle=\color{RoyalBlue},%\bfseries,
    basicstyle=\small\ttfamily,
    %identifierstyle=\color{NavyBlue},
    commentstyle=\color{Green}\ttfamily,
    stringstyle=\rmfamily,
    numbers=none,%left,%
    numberstyle=\scriptsize,%\tiny
    stepnumber=5,
    numbersep=8pt,
    showstringspaces=false,
    breaklines=true,
    frameround=ftff,
    frame=single,
    belowcaptionskip=.75\baselineskip
    %frame=L
} 
\lstdefinestyle{python}{language=Python,
    keywordstyle=\color{RoyalBlue},%\bfseries,
    basicstyle=\tiny\ttfamily,
    %identifierstyle=\color{NavyBlue},
    commentstyle=\color{Green}\ttfamily,
    stringstyle=\rmfamily,
    numbers=none,%left,%
    numberstyle=\scriptsize,%\tiny
    stepnumber=5,
    numbersep=8pt,
    showstringspaces=false,
    breaklines=true,
    frameround=ftff,
    frame=single,
    belowcaptionskip=.75\baselineskip
    %frame=L
} 



\usepackage{tikz}
\usetikzlibrary{fit,shapes.misc}

\newcommand\marktopleft[1]{%
    \tikz[overlay,remember picture] 
    \node (marker-#1-a) at (0,1.5ex) {};%
}
\newcommand\markbottomright[2][red]{%
    \tikz[overlay,remember picture] 
    \node (marker-#2-b) at (0,0) {};%
    \tikz[overlay,remember picture,thick,inner sep=3pt,fill=red]
    \node[draw,rectangle,fill=#1,nearly transparent,fit=(marker-#2-a.center) (marker-#2-b.center)] {};%
}

%\title{\bf my title}
%\author{\bf Me \\ and others}
\institute{\large{L3S Research Center / Leibniz University of Hanover\\ Hanover, Germany}}
%\date{\today}

\begin{document}
\begin{frame}
\frametitle{Implicit Feedback}

\begin{table}[t]
\begin{tabular}{c|cccccc}
    &Anna&Berta&Claudia&Dagmar\\\hline
    The Shawshank Redemption&1&&1&\\
    The Godfather&&1&1&\\
    The Godfather: Part II&&1&&1\\
    Pulp Fiction&1&1&&1\\
    The Good, the Bad and the Ugly&1&&1&\\
\end{tabular}
\end{table}
% Consider the following scenario where we have a viewing history of some persons as illustrated here. [vorlesen]
% We also call this implicit feedback because the women weren't ask to rate the movies
% they just watched them and we're trying to get some useful information out of it.
% The item prediction task a recommender has to solve is to predict which movies are interesting for which persons
\end{frame}

\frame{\titlepage} % overview over the thesis with reference to recsyslab as major contribution.
%\frame{\tableofcontents[currentsection]}
\frame{
    \frametitle{Contents}
    \tableofcontents
   % [pausesections]
}

\section{Item Prediction Task and Implicit Feedback}
\frame{
    \frametitle{Contents}
    \tableofcontents[currentsection]
   % [pausesections]
}
\begin{frame}
\frametitle{Implicit Feedback}

\begin{table}[t]
\begin{tabular}{c|cccccc}
    &Anna&Berta&Claudia&Dagmar\\\hline
    The Shawshank Redemption&1&&1&\\
    The Godfather&&1&1&\\
    The Godfather: Part II&&1&&1\\
    Pulp Fiction&1&1&&1\\
    The Good, the Bad and the Ugly&1&&1&\\
\end{tabular}
\end{table}
% Consider the following scenario where we have a viewing history of some persons as illustrated here. [vorlesen]
% We also call this implicit feedback because the women weren't ask to rate the movies
% they just watched them and we're trying to get some useful information out of it.
% The item prediction task a recommender has to solve is to predict which movies are interesting for which persons
\end{frame}
\begin{frame}
\frametitle{Item Prediction Task}

\begin{table}[t]
\begin{tabular}{c|cccccc}
    &Anna&Berta&Claudia&Dagmar\\\hline
    The Shawshank Redemption&1&&1&?\\
    The Godfather&&1&1&?\\
    The Godfather: Part II&&1&&1\\
    Pulp Fiction&1&1&&1\\
    The Good, the Bad and the Ugly&1&&1&?\\
\end{tabular}
% So for example when dagmar goes to the cinema, which movie should we recommend her?
% Here it's quite probably The Godfather to she should see, but normally those systems
% are used in scenarios with thousands of persons and movies.
\end{table}

\end{frame}
\begin{frame}

\frametitle{Notation}

\begin{table}[t]
    \begin{tabular}{c|cccccc}
        &\marktopleft{c2}Anna&Berta&Claudia&Dagmar\markbottomright[red]{c2}\\\hline
        \marktopleft{c3}The Shawshank Redemption&\marktopleft{c1}1&&1&\\
        The Godfather&&\marktopleft{c4}1&1&\\
        The Godfather: Part II&&1&&1\\
        Pulp Fiction&1&1\markbottomright[magenta]{c4}&&1\\
        The Good, the Bad and the Ugly\markbottomright[green]{c3}&1&&1&\markbottomright[blue]{c1}\\
\end{tabular}
\end{table}
%From now on I will refer to those with:
\textcolor{green}{Items}\\
\textcolor{red}{Users}\\
\textcolor{blue}{Interactions}\\
\textcolor{magenta}{Basket of $u$}
% Please note, that it is not important what exactly they are.
% Interactions can be purchases, views, clicks etc
% and then the items will be for example goods to buy or videos.
\end{frame}

\begin{frame} 
\frametitle{Related Work}
\begin{itemize} 
    \item \textbf{MyMediaLite} C\#, several recommender algorithms
    \item \textbf{PREA} (Personalized Recommandation Algorithms Toolkit) Java, recommender algorithms and evaluation metrics
    \item \textbf{Apache Mahout} Java, machine learning on top of Apache Hadoop
    \item \textbf{Duine Framework} Java, combination of multiple recommender algorithms
    \item \textbf{Cofi} C++, centers on one recommender algorithm
    \item \textbf{Lenskit} Java, toolkit for recommendation and evaluation
    \item \textbf{Scikit-learn} Python, machine learning, no recommender algorithms
\end{itemize}
\end{frame}
\section{Recsyslab}
\frame{
    \frametitle{Contents}
    \tableofcontents[currentsection]
   % [pausesections]
}
\begin{frame} 
\frametitle{Motivation for recsyslab}\pause
\begin{itemize}
    \item Python for easy readable source code\pause
    \item Simple to use\pause
    \item State of the art recommender algorithms\pause
    \item Open source license: GPLv3
\end{itemize}
\end{frame}
\begin{frame}
\frametitle{General Structure}
\begin{centering}
    \includegraphics[page=1, scale=0.34]{packagediagram.pdf}
%    \hspace*{8cm}
\end{centering}
\end{frame}
\begin{frame}[fragile]
    \frametitle{Get recsyslab}
%    Github repository:\\
    \url{github.com/Foolius/recsyslab}\\
    \hspace*{8cm}\\
%    Direct download of a .zip:\\
    \url{github.com/Foolius/recsyslab/archive/master.zip}\\
    \hspace*{8cm}\\
%    Clone it with git:
    \begin{lstlisting}[style=pseudocode]
$ git clone 
    https://github.com/Foolius/recsyslab.git
    \end{lstlisting}
\end{frame}

\begin{frame}
    \frametitle{Recommender Algorithms in recsyslab}\pause
    \begin{itemize}
        \item Matrix Factorization~\cite{matrixfactorization}
            \begin{itemize}
                \item Bayesian Personalized Ranking (BPRMF)~\cite{Rendle:2009:BBP:1795114.1795167}
                \item RankMFX~\cite{diaz2012happening}
                \item Ranking SVD~\cite{jahrer2011collaborative}
            \end{itemize}\pause
        \item k-Nearest-Neighbor\cite{Karypis:2001:EIT:502585.502627}
            \begin{itemize}
                \item Item-Based
                \item User-Based
            \end{itemize}\pause
		  \item Slope One~\cite{DBLP:journals/corr/abs-cs-0702144}\pause
        \item Non-Personalized
            \begin{itemize}
                \item Constant
                \item Random
            \end{itemize}
    \end{itemize}

\end{frame}
%\begin{frame}
%\frametitle{Non-Personalized}
%\begin{description}
%    \item[Constant] Recommend the most popular items
%    \item[Random] Recommend randomly chosen items
%\end{description}
%\end{frame}
\begin{frame}
    \frametitle{Matrix Factorization~\cite{matrixfactorization}}
\begin{table}[t]
\begin{tabular}{c|cccccc}
    &Anna&Berta&Claudia&Dagmar\\\hline
    The Shawshank Redemption&\marktopleft{c9}1&0&1&0\\
    The Godfather&0&1&1&0\\
    The Godfather: Part II&0&1&0&1\\
    Pulp Fiction&1&1&0&1\\
    The Good, the Bad and the Ugly&1&0&1&0\markbottomright[green]{c9}\\
\end{tabular}
\end{table}
\begin{center}
Find $W$ and $H$ so:
\(\textcolor{green}{\hat{M}} = W\;H^\top\).\\
        \begin{equation}
    \text{Score}(u,i) = W_u I_i^\top.
\end{equation}
\end{center}
%explain the transpose
%explain that W and H are features of the items and the features
\end{frame}
\begin{frame}[fragile]
    \frametitle{Matrix Factorization, Training}
\begin{lstlisting}[style=pseudocode]
repeat until conversion of W and H:
    u=randomly chosen user
    i=randomly chosen item U interacted with
    j=randomly chosen item U did not interact 
      with

    X=H[i] - H[j]
    wx=dot product of W[u] and X
    dloss=(derivative of the loss 
           function of wx and 1) * learningRate

    W[u]+=dloss*(H[i] - H[j]) #These three lines
    H[i]+=dloss*W[u]          #have to be
    H[j]+=dloss*-W[u]         #executed at once
\end{lstlisting}
\end{frame}

\begin{frame}[fragile]
%\frametitle{Matrix Factorization, Training}
\beamertemplatenavigationsymbolsempty
\begin{lstlisting}[style=python]
for i in xrange(0, iterations):
    u = random.choice(R.keys())
    userItems = [x[0] for x in R[u]]
    # the positive example
    i = userItems[np.random.random_integers(0, len(userItems) - 1)]
    # the negative example
    j = np.random.random_integers(0, m_items)
    # if  j is also relevant for u we continue
    # we need to see a negative example to contrast the positive one
    while j in userItems:
        j = np.random.random_integers(0, m_items)

    X = H[i] - H[j]
    wx = np.dot(W[u], X)
    dloss = dlossF(wx, y)

    # temp
    wu = W[u]
    hi = H[i]
    hj = H[j]
    if dloss != 0.0:
        # Updates
        eta_dloss = learningRate * dloss
        W[u] += eta_dloss * (hi - hj)
        H[i] += eta_dloss * wu
        H[j] += eta_dloss * (-wu)
        W[u] *= scaling_factorU
        H[i] *= scaling_factorI
        H[j] *= scaling_factorJ
\end{lstlisting}
\end{frame}
\begin{frame}
    \frametitle{k-Nearest-Neighbor~\cite{Karypis:2001:EIT:502585.502627}}
    \begin{enumerate}
        \item <2->Compute similarity of each (item, item) pair
            \begin{equation}
                \text{sim}(i,j) = \cos(\vec{i}, \vec{j})=\frac{\vec{i} \cdot \vec{j}}{||\vec{i}||_{2} ||\vec{j}||_{2}}
            \end{equation}
    \item <3->For each item, save the $k$ items with the highest similarity\\(= neighbors)
    \item <4->Compute the union of the neighbors of the basket of $u$
    \item <5->For each item in this set compute the sum of similarities to the basket of $u$
    \item <6->Sort by this score and return the first $N$ items
\end{enumerate}
% This is item based but it can also be done user based.
\end{frame}
\begin{frame}
\frametitle{k-Nearest-Neighbor~\cite{Karypis:2001:EIT:502585.502627}}
\begin{table}[t]
\begin{tabular}{c|cccccc}
    &Anna&Berta&Claudia&Dagmar\\\hline
    The Shawshank Redemption&\marktopleft{c5}1&0&1&0\markbottomright[red]{c5}\\
    The Godfather&0&1&1&0\\
    The Godfather: Part II&\marktopleft{c7}0&1&0&1\markbottomright[green]{c7}\\ 
    Pulp Fiction&1&1&0&1\\
    The Good, the Bad and the Ugly&1&0&1&0\\
\end{tabular}
\end{table}
\begin{equation*}
    \text{sim}(i,j) = \cos(\vec{i}, \vec{j})=\frac{\vec{i} \cdot \vec{j}}{||\vec{i}||_{2} ||\vec{j}||_{2}}
    = \frac{0}{2}=0
\end{equation*}
\end{frame}
\begin{frame}
\frametitle{k-Nearest-Neighbor~\cite{Karypis:2001:EIT:502585.502627}}
\begin{table}[t]
\begin{tabular}{c|cccccc}
    &Anna&Berta&Claudia&Dagmar\\\hline
    The Shawshank Redemption&1&0&1&0\\
    The Godfather&0&1&1&0\\
    The Godfather: Part II&\marktopleft{c7}0&1&0&1\markbottomright[green]{c7}\\
    Pulp Fiction&\marktopleft{c8}1&1&0&1\markbottomright[blue]{c8}\\
    The Good, the Bad and the Ugly&1&0&1&0\\
\end{tabular}
\end{table}
\begin{equation*}
    \text{sim}(i,j) = \cos(\vec{i}, \vec{j})=\frac{\vec{i} \cdot \vec{j}}{||\vec{i}||_{2} ||\vec{j}||_{2}}
    =\frac{2}{\sqrt{2}\sqrt{3}}
\end{equation*}
% go back again and explain the rest of the algorithm.
\end{frame}
%\begin{frame}
%\frametitle{Slope One~\cite{DBLP:journals/corr/abs-cs-0702144}}
%\begin{table}[t]
%\begin{tabular}{c|cccccc}
%    &Anna&Berta&Claudia&Dagmar\\\hline
%    The Shawshank Redemption&1&&1&\\
%    The Godfather&&1&1&\\
%    The Godfather: Part II&&1&&1\\
%    Pulp Fiction&\marktopleft{c10}1&1&2&1\markbottomright[green]{c10}\\
%    The Good, the Bad and the Ugly&\marktopleft{c11}2&?&1&3\markbottomright[red]{c11}\\
%\end{tabular}%database changed so slope one works.
%\end{table}
%\begin{equation}
%
%\end{equation}
%\end{frame}
%should i include slopeone? it doesn't work great and is complicated to explain.

\begin{frame}
    \frametitle{Leave-one-out Protocol~\cite{leaveoneout}}\pause
    \begin{enumerate}
        \item Randomly choose one interaction per user and hide them\pause
        \item Train the recommender system with the remaining interactions\pause
        \item Get recommendations for every user\pause
        \item Compute the chosen evaluation metric with the hidden items and the recommendations
    \end{enumerate}

\end{frame}
%\subsection{Evaluation Metrics}
\begin{frame}
    \frametitle{Evaluation Metrics in recsyslab}
    \begin{itemize}
        \item Hitrate/Recall@N~\cite{Karypis:2001:EIT:502585.502627, Sarwar00applicationof}
        \item Precision~\cite{Sarwar00applicationof}
        \item F1~\cite{Sarwar00applicationof}
        \item Mean Reciprocal Hitratea (MRHR)~\cite{DBLP:conf/icdm/NingK11}
        \item Area under the ROC (AUC)~\cite{Rendle:2009:BBP:1795114.1795167}
    \end{itemize}
\end{frame}

\begin{frame} 
\frametitle{Hitrate/Recall@N~\cite{Karypis:2001:EIT:502585.502627, Sarwar00applicationof}}
\begin{equation} 
\text{Recall@N}=\frac{\sum_{u \in U} H_u \cap \text{topN}_u}{|H|}
\end{equation}\\
\vspace{6.4mm}
\begin{description}
    \item[$H$] hidden interactions\\
    \item[$H_u$] the hidden interaction of $u$\\
    \item[$U$] set of users
    \item[$topN_u$] $N$ recommendations for $u$
\end{description}
\end{frame}

%\begin{frame} 
%\frametitle{Precision~\cite{Sarwar00applicationof}}
%\begin{equation} 
%\text{Precision}=\frac{\sum_{u \in U} H_u \cap \text{topN}_u}{N \times |U|}
%\end{equation}\\
%\vspace{6.4mm}
%\begin{description}
%    \item[$H$] hidden interactions\\
%    \item[$H_u$] the hidden interaction of $u$\\
%    \item[$U$] set of users
%    \item[$topN_u$] $N$ recommendations for $u$
%\end{description}
%\end{frame}
%
%\begin{frame} 
%    \frametitle{F1~\cite{Sarwar00applicationof}}
%\begin{equation} 
%\text{F1}=\frac{2 \times \text{Recall@N} \times 
%\text{Precision}}{\text{Recall@N} + \text{Precision}}.
%\end{equation}\\
%\vspace{6.4mm}
%\begin{description}
%    \item[$H$] hidden interactions\\
%    \item[$H_u$] the hidden interaction of $u$\\
%    \item[$U$] set of users
%    \item[$topN_u$] $N$ recommendations for $u$
%\end{description}
%\end{frame}
%
%\begin{frame} 
%    \frametitle{Mean Reciprocal Hitrate~\cite{DBLP:conf/icdm/NingK11}}
%\begin{equation} 
%\text{MRHR}=\frac{1}{|U|} \sum_{u \in U} \frac{1}{\text{pos}(\text{topN}_{u},H_{u})},
%\end{equation}\\
%\vspace{6.4mm}
%\begin{description}
%    \item[$H$] hidden interactions\\
%    \item[$H_u$] the hidden interaction of $u$\\
%    \item[$U$] set of users
%    \item[$topN_u$] $N$ recommendations for $u$
%    \item[$\text{pos}(\text{topN}_{u},H_{u})$] position of the hidden item in the list of recommendations
%\end{description}
%\end{frame}
%
%\begin{frame} 
%    \frametitle{Area under the ROC (AUC)~\cite{Rendle:2009:BBP:1795114.1795167}}
%\begin{equation} 
%\text{AUC}=\frac{1}{|U|}\sum_{u \in U} \frac{1}{|E(u)|} 
%\sum_{(i,j) \in E(u)} \delta(x_{ui}>x_{uj}),
%\end{equation}\\
%\begin{equation}
%\delta(x)=\begin{cases}1, & \text{if x is true}, \\
%                       0, & \text{otherwise.}
%\end{cases}
%\end{equation}
%\begin{equation}
%E(u) =\{(i,j)|(u,i) \in H \land (u,j) \not\in (H \cup T)\}.
%\end{equation}
%%\vspace{6.4mm}
%\begin{description}
%    \item[$H$] hidden interactions\\
%    \item[$H_u$] the hidden interaction of $u$\\
%    \item[$U$] set of users
%    \item[$x_{ui}$] predicted score of the interaction between $u$ and $i$
%\end{description}
%\end{frame}

\section{Demonstration of recsyslab}
\frame{
    \frametitle{Contents}
    \tableofcontents[currentsection]
   % [pausesections]
}

\begin{frame}
    \frametitle{MovieLens Dataset}
    \begin{itemize}
        \item 100 000 interactions
        \item 943 users
        \item 1682 items
        \item Text file with columns separated with a single tab
        \item Provided by the GroupLens research lab
    \end{itemize}
\end{frame}
\begin{frame}[fragile]
    \frametitle{Movielens Dataset Excerpt}
\begin{lstlisting}
user    item    rating  time stamp
196     242     3       881250949
186     302     3       891717742
22      377     1       878887116
244     51      2       880606923
166     346     1       886397596
298     474     4       884182806
115     265     2       881171488
253     465     5       891628467
305     451     3       886324817
6       86      3       883603013
62      257     2       879372434
286     1014    5       879781125
200     222     5       876042340
\end{lstlisting}
\end{frame}

\section{Outlook \& Conclusions}
\frame{
    \frametitle{Contents}
    \tableofcontents[currentsection]
   % [pausesections]
}
\begin{frame}
\frametitle{Outlook}\pause
\begin{itemize}
	\item More algorithms\pause
	\item Refine hyperparameters with more experiments\pause
	\item Update function for the models\pause
	\item Incorporate user feedback about the library
\end{itemize}
\end{frame}

\begin{frame} 
\frametitle{Conclusions} 
\begin{itemize}
    \item Python for easy readable source code \visible<2->{$\checkmark$}
    \item Simple to use \visible<3->{$\checkmark$}
    \item State of the art recommender algorithms \visible<4->{$\checkmark$}
    \item Open source license: GPLv3 \visible<5->{$\checkmark$}
\end{itemize}
\end{frame}

\begin{frame}[fragile]
    \frametitle{Have Fun!}
%    Github repository:\\
    \url{github.com/Foolius/recsyslab}\\
    \hspace*{8cm}\\
%    Direct download of a .zip:\\
    \url{github.com/Foolius/recsyslab/archive/master.zip}\\
    \hspace*{8cm}\\
%    Clone it with git:
    \begin{lstlisting}[style=pseudocode]
$ git clone 
    https://github.com/Foolius/recsyslab.git
    \end{lstlisting}
\end{frame}

\bibliographystyle{apalike}
%\bibliographystyle{alphaurl}
%\bibliographystyle{plainnat}
\bibliography{Bibliography}
%\frame{\titlepage}
%
%
%%=====================
%\section[Outline]{}
%\begin{frame}
%\frametitle{Outline}
%%
%\tableofcontents
%%
%\end{frame}
%
%
%%=====================
%\section{Section 1}
%
%\begin{frame}
%		\begin{itemize}
%				\item one
%				\item two	
%		\end{itemize}
%\end{frame}
%
%% --------------------------------------------
%%
%
%\section{Section 2}
%\begin{frame}
%	\frametitle{Box}
%	\begin{alertblock}{alert box title}
%		\begin{center}
%			Content in Alert Box :)
%			\begin{itemize}
%				\item one
%				\item two
%			content...
%			\end{itemize}
%		\end{center}
%	\end{alertblock}
%\end{frame}

\end{document}
%
