\documentclass{beamer}
\usepackage{amsmath,amsfonts,amssymb}
\setbeamertemplate{footline}[frame number]
\usetheme{Warsaw}  %% Themenwahl
\usepackage{color}

\title{Multi-purpose Library of Recommender System Algorithms for the Item Prediction Task}
\subtitle{Presentation of my Bachelor Thesis}
\author{Julius Kolbe}
\institute{Fakult\"at f\"ur Elektrotechnik und Informatik\\Institut f\"ur Verteilte Systeme}
\date{\today}



\usepackage{tikz}
\usetikzlibrary{fit,shapes.misc}

\newcommand\marktopleft[1]{%
    \tikz[overlay,remember picture] 
    \node (marker-#1-a) at (0,1.5ex) {};%
}
\newcommand\markbottomright[2][red]{%
    \tikz[overlay,remember picture] 
    \node (marker-#2-b) at (0,0) {};%
    \tikz[overlay,remember picture,thick,inner sep=1pt,fill=red]
    \node[draw,rectangle,fill=#1,nearly transparent,fit=(marker-#2-a.center) (marker-#2-b.center)] {};%
}

\begin{document}
\frame{\titlepage} % overview over the thesis with reference to recsyslab as major contribution.
%\frame{\tableofcontents[currentsection]}
\frame{
    \frametitle{Contents}
    \tableofcontents
   % [pausesections]
}

\section{Background}
\frame{
    \frametitle{Contents}
    \tableofcontents[currentsection]
   % [pausesections]
}
\subsection{Item Prediction Task and Implicit Feedback}
\begin{frame}
\frametitle{Implicit Feedback}

\begin{table}[t]
\begin{tabular}{c|cccccc}
    &Anna&Berta&Claudia&Dagmar\\\hline
    The Shawshank Redemption&1&&1&\\
    The Godfather&&1&1&\\
    The Godfather: Part II&&1&&1\\
    Pulp Fiction&1&1&&1\\
    The Good, the Bad and the Ugly&1&&1&\\
\end{tabular}
\end{table}
% Consider the following scenario where we have a viewing history of some persons as illustrated here. [vorlesen]
% We also call this implicit feedback because the women weren't ask to rate the movies
% they just watched them and we're trying to get some useful information out of it.
% The item prediction task a recommender has to solve is to predict which movies are interesting for which persons
\end{frame}
\begin{frame}
\frametitle{Item Prediction Task}

\begin{table}[t]
\begin{tabular}{c|cccccc}
    &Anna&Berta&Claudia&Dagmar\\\hline
    The Shawshank Redemption&1&&1&?\\
    The Godfather&&1&1&?\\
    The Godfather: Part II&&1&&1\\
    Pulp Fiction&1&1&&1\\
    The Good, the Bad and the Ugly&1&&1&?\\
\end{tabular}
% So for example when dagmar goes to the cinema, which movie should we recommend her?
% Here it's quite probably The Godfather to she should see, but normally those systems
% are used in scenarios with thousands of persons and movies.
\end{table}

\end{frame}
\begin{frame}

\frametitle{Notation}

\begin{table}[t]
    \begin{tabular}{c|cccccc}
        &\marktopleft{c2}Anna&Berta&Claudia&Dagmar\markbottomright[red]{c2}\\\hline
        \marktopleft{c3}The Shawshank Redemption&\marktopleft{c1}1&&1&\\
        The Godfather&&1&1&\\
        The Godfather: Part II&&1&&1\\
        Pulp Fiction&1&1&&1\\
        The Good, the Bad and the Ugly\markbottomright[green]{c3}&1&&1&\markbottomright[blue]{c1}\\
\end{tabular}
\end{table}
%From now on I will refer to those with:
\textcolor{green}{Items}\\
\textcolor{red}{Users}\\
\textcolor{blue}{Interactions}
% Please note, that it is not important what exactly they are.
% Interactions can be purchases, views, clicks etc
% and then the items will be for example goods to buy or videos.

\end{frame}
\subsection{Evaluation}
\begin{frame}
    \frametitle{Leave-one-out Protocol}
    \begin{enumerate}
        \item Randomly choose one interaction per user and hide them
        \item Train the recommender system with the remaining interactions
        \item Get recommendations for every user
        \item Compare the recommendations with the hidden items
    \end{enumerate}

\end{frame}
\begin{frame} 
\frametitle{Evaluation Metrics}
\end{frame}

\section{Recommendation Algorithms}
\frame{
    \frametitle{Contents}
    \tableofcontents[currentsection]
   % [pausesections]
}
\begin{frame}
\frametitle{Non-Personalized}
\end{frame}
\begin{frame}
\frametitle{k-Nearest-Neighbor}
\end{frame}
\begin{frame}
\frametitle{Matrix Factorization}
\end{frame}
\begin{frame}
\frametitle{Slope One}
\end{frame}

\section{Recsyslab}
\frame{
    \frametitle{Contents}
    \tableofcontents[currentsection]
   % [pausesections]
}
\begin{frame} 
\frametitle{Motivation} %also comparison with other frameworks
\end{frame}
\begin{frame}
\frametitle{General Structure}
\end{frame}
\end{document}
