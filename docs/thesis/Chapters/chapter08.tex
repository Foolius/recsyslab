
\chapter{Conclusions}
In this document I provided
\begin{itemize}
\item an overview over the research area
\item explanations of the implemented recommender algorithms
\item explanations of the implemented test metrics
\item a user manual explaining how to use every recommonder algorithm and test metric
\item explanations of the internal structure of recsyslab
\item test results
\end{itemize}

While recsyslab provides
\begin{itemize}
\item several state of the art recommender algorithms
\item several widely used test metrics
\item a simple infrastructure to use these together
\item easy extendable with new recommender algorithms or test metrics
\item acceptable test results
\end{itemize}
After reading this document every beginner should be able to use and understand recsyslab.
We hope this will help students start working in this field and researchers to compare their new algorithms
with already existing algorithms without wasting much time implementing something else
then their own algorithms.


\section{Outlook}
For future work we could
\begin{itemize}
\item implement more recommender algorithms
\item implement more test metrics
\item visualize test results graphically
\item update the model with new interactions so we don't have to train the model from ground up again
\end{itemize}

But apart of that we hope to get feedback from users to improve recsyslab.
It will be interesting to see in which parts students have problems to 
understand what's going on. When such parts are discovered we want to make
them easier to understand. Also which new features, test metrics and recommender algorithms
get implemented will depend on the user feedback. For the recommender algorithms
we will observe the research in this area and implement new popular recommender algorithms
to keep recsyslab up to date.
