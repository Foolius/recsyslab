\chapter{Experiments}
\label{experiments}
In this chapter we will show test results of every metric with every
recommender algorithm and how they were computed.


\section{Execution}
The results were computed using the recommender algorithms and 
the test metrics implemented in recsyslab. It was done 
like it is shown in the user manual in~\ref{usermanual}.
Only difference is that for simplification the different
getRec methods and metric methods were each in a list and the 
algorithm is iterating trough both in two nested for loops
to guarantee that every algorithm is tested with every metric.
The code which generated the table in \ref{results} is in 
the experiments.py file in the bin directory of recsyslab.
So check it out for further implementation details.


\section{Results}
\label{results}

\begin{tabular}{rlllll} \toprule
    recommender  & hitrate & precision & f1 & mrhr & auc \\ \midrule
    constant & 0.0731 & 6.9 & 0.1448 & 0.0207 & 0.8263 \\
    randomRec & 0.0053 & 0.5 & 0.0104 & 0.0011 & 0.4790 \\
    itemKnn & 0.2576 & 24.3 & 0.5099 & 0.1151 & 0.8687 \\
    userKnn & 0.2619 & 24.7 & 0.5183 & 0.1203 & 0.9279 \\
    BPRMF& 0.2990 & 28.2 & 0.5918 & 0.1261 & 0.9434 \\
    RankMFX & 0.1389 & 13.1 & 0.2749 & 0.0538 & 0.7447 \\
    Ranking SVD & 0.0084 & 0.8 & 0.0167 & 0.0030 & 0.5503 \\
    slopeone & 0.0 & 0.0 & 0 & 0.0 & 0.6695 \\ \bottomrule
\end{tabular}

\section{Comparison}
To show the performance of recsyslab we compare our test results with results 
from this paper:~\cite{deshpande2004item}


\begin{tabular}{rllll} \toprule
 & \multicolumn{2}{c}{hitrate} & \multicolumn{2}{c}{mrhr} \\ \cmidrule(r){2-3} \cmidrule(r){4-5}
 & recsyslab & \cite{deshpande2004item} & recsyslab & \cite{deshpande2004item} \\ \midrule
    constant & 0.0731 & 0.131 & 0.0207 & 0.046 \\
    itemKnn & 0.2576 & 0.271 & 0.1151 & 0.119 \\
    userKnn & 0.2619 & 0.281 & 0.1203 & 0.128 \\ \bottomrule
\end{tabular}

It can be seen that recsyslab performs almost as good as the reference.
The small advantage probably comes from normalization we did not use when
we were computing these results.
