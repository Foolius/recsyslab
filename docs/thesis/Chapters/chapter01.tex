\chapter{Introduction}
Recommender systems are widely used in e-commerce and are an active field of research; new and refined algorithms are proposed continuously. Easy and comprehensive evaluation tools are required to assess their performance in a systematic way.  

With the simple syntax and the interactivity
of Python, \textit{recsyslab} which is available at

\vspace{2mm}
\centerline{\url{https://github.com/Foolius/recsyslab/archive/master.zip}}
\vspace{2mm}is designed in such a way, so as to facilitate experimentation with different recommender systems
algorithms. Our goal is to provide beginners and experts with an extensible tool to evaluate classic and modern algorithms, to comprehend them, and to extend the existing core implementation if necessary. With this document, recommender systems' enthusiasts should be able to get started in trying out different recommender algorithms
without needing any previous knowledge or long configuration.


\section{What a Recommender System does?}
A Recommender System works in a scenario with users, items, and interactions
between these two. Such a scenario could be an online shop such as \url{amazon.com},
where the interactions are purchases of items by users or a video
platform, where the users interact with items (videos) by watching
them, e.g., \url{youtube.com}. Based on the past interactions of the users
a recommender system presents them with a ranked list of items, which meets their interest and taste.

Two fundamental tasks for recommender systems are:
\begin{itemize}
    \item \textbf{Rating Prediction}:
When the feedback is provided explicitly through ratings (e.g., stars in a 1 to 5 scale), the scenario is called rating prediction, and the task consists of predicting the real value of the rating. This task can be cast as a regression problem.
  
The rating prediction task gained popularity when Netflix announced a prize~\cite{netflixprize}
in September 2009, offering a one-million-dollar prize to a team able to improve Netflix's existing recommender system.

    \item \textbf{Item Prediction}:
When the interactions are implicit, like, in case of purchases or clicks, then the problem can be cast as item prediction or top-N recommendation. In this case, the task is to recommend a short list of items to meet the particular taste of the users. Please note that in this scenario we are not interested in regressing a value for a rating, but in relative ranking of items that meet the user's individual information needs. 

\end{itemize}
In this work, the focus lies precisely on this scenario of implicit feedback and item prediction. The algorithms implemented within recsyslab compute a score which is used to rank the items relative to each other, but this score is not an absolute rating.

Note that ratings can also be interpreted as the strength for implicit feedback. Some algorithms implemented
in this library make use of this information, but none of them explicitly predict ratings, unlike the usual practice in rating prediction scenarios. Rating information is used to derive a relative order within the items, that is, it is sufficient to only have a ranked list of items according to the preferences of the users. This information is then used to train our item prediction algorithms. 


\section{Contributions}
The contributions of this bachelor thesis are:
\begin{enumerate}
    \item the implementation of an easy to use and Open Source library: \textit{recsyslab}, which is released under the GNU GPLv3\cite{gpl} and includes a set of the most important state-of-the-art recommender system algorithms for item prediction. Special care has been given to the source code produced, to keep it simple and readable, and as close as possible to the pseudocode of the algorithms.
    \item supporting infrastructure for testing recommender algorithms
    %\item easy readable source code
    \item extensive user manual
    %\item test results of the algorithms computed with recsyslab
    \item explanations of the technical background and software design
\end{enumerate}


\section{Structure of this Document}
The next chapter includes an overview of the research area in recommender systems,
together with explanations of different evaluation metrics and the Leave-one-out
protocol commonly used to evaluate top-N recommender algorithms.

After that, we provide short descriptions of already existing 
projects which provide something similar to recsyslab (Chapter~\ref{relatedwork}).

In Chapter \ref{recommendationalgorithms}, we present the implemented recommender
algorithms along with the core of their implementation. 

After that, we show the inner structure of recsyslab
to make it easy for developers to extend or change the library (Chapter~\ref{design}).

Chapter \ref{usermanual} provides an extensive guide for recsyslab explaining
how the library is used.

The last chapter includes conclusions and an outlook to future work.
