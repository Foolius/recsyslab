\chapter{Introduction}


\section{Motivation}

The library together with this document shall provide a "cookbook"
for recommender systems. Recommender systems are widely used in ecommerce
and are an active field of research so an easy accessible library will be 
useful in the future.
With the simple syntax and the interactivity
of Python recsyslab is aimed at beginners to simply experiment with different
algorithms and comprehend them. Together with this document everybody
should be able to get started to try out different recommender algorithms
without needing any previous knowledge or long configuration.


\section{Task (what a Recommender System does)}

A Recommender System works in a scenario with users, items and interactions
between these two. Such a scenario could be an online shop,
where the interactions are purchases of items by users or a video
platform, where the users interact with items (videos) by watching
them e.g. youtube.com~\cite{youtube}. Based on the past interactions of the users
a Recommender System presents them with a ranked list of items, which meets their interest.

Two common scenarios are:
\begin{itemize}
    \item Rating Prediction:
When the feedback is provided explicitly like ratings the scenario is called rating prediction. 
The rating prediction task got very popular when netflix announced a prize~\cite{netflixprize}
in September 2009. The algorithm which performs best in predicting ratings 
would be granted the prize money.
    \item Item Prediction:
When the interactions are implicit like purchases or clicks, then the
scenario is called item prediction. But in this work the focus lies on implicit feedback or item prediction.
However ratings can also be interpreted as the strength of implicit feedback.
For example how often a user purchased an item. Some algorithms implemented
in this library can use this information but none will explicitly
predict ratings like it is usual in rating prediction scenarios because
it is normally not useful to predict explicit absolute ratings when
you want to propose items a user could find interesting. In this case
it is sufficient to only have a ranked list of items according to the 
preferences of the users. Most of the recommender algorithms in recsyslab
compute a score which is used to rank the items relative to each other.
But this score is not an absolute rating.
\end{itemize}


\section{Contributions}

\begin{enumerate}
    \item recsyslab with state of the art algorithms
    \item supporting infrastructure for testing recommender algorithms
    \item easy to use library
    \item easy readable source code
    \item extensive user manual
    \item test results of the algorithms computed with recsyslab
    \item explanations of the technical background
\end{enumerate}


\section{Structure of this Document}
In the next chapter is an overview over the research area of recommender systems
together with explanations of different evaluation metrics and the Leave-one-out
Protocol used to evaluate recommender algorithms.
After that we provide short descriptions over already existing 
projects providing something similar to recsyslab.
In chapter \ref{recommendationalgorithms} we present the implemented recommender
algorithms. Followed by a chapter with results and explanations about how the
results have been obtained. After that we will show the inner structure of recsyslab
to make it easy for developer to extend or change the library.
In chapter \ref{usermanual} is an extensive guide for recsyslab explaining
how the library is used.
In the last chapter are the conclusions and an outlook to future work.



