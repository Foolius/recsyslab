\chapter{Introduction}
\automark[section]{chapter}


\section{Motivation}

The library together with this document shall provide a "cookbook"
for recommender systems. With the simple syntax and the interactivity
of Python it is aimed at beginners to simply experiment with different
algorithms. Especially the interactivity is missing in the already
existing libraries because none of them is written in Python.


\section{Task (what a Recommender System does)}

A Recommender System works in a scenario with users, items and interactions
between these two. Such a scenario could be an online shop,
where the interactions are purchases of items by users or a video
platform, where the users interact with items (videos) by watching
them e.g. youtube.com~\cite{youtube}. Based on the past interactions of the users
a Recommender System presents them with a ranked list of items, which meets their interest.

The interactions can be implicit like purchases or clicks, then the
scenario is also called item prediction. When the feedback is provided
explicit like ratings the scenario is called rating prediction. In
this work the focus lies on implicit feedback or item prediction.
However ratings can be interpreted as the strength of implicit feedback.
For example how often a user purchased an item. Some algorithms implemented
in this library can use this information but none will explicitly
predict ratings like it's usual in rating prediction scenarios.


\section{Contributions}

The main contribution of my work is the interactive library I wrote~\cite{recsyslab}. 
Also in this document I provide explanations about
the algorithms implemented in the library and an extensive user manual
of the library.

