\chapter{Appendix - Docstrings}
In this chapter we provide the inline documentation which is also included
in recsyslab.

\begin{lstlisting}
Help on package bin:

NAME
    bin - Package with the Python scripts of the recsyslab toolbox.

FILE
    /home/jk/recsyslab/bin/__init__.py

DESCRIPTION
    It contains the Packages:
        recommender -- Package for the recommender algorithms
        util        -- Package with all the other stuff needed to try out the
                        recommender algorithms

PACKAGE CONTENTS
    experiments
    main
    manual
    recommender (package)
    testing
    usecases
    util (package)
\end{lstlisting}

\begin{lstlisting}
Help on package bin.recommender in bin:

NAME
    bin.recommender - Package containing the recommending algorithms.

FILE
    /home/jk/recsyslab/bin/recommender/__init__.py

DESCRIPTION
    These are:
        nonpersonalized --  constant and random
        knn             --  k-Nearest-Neigbor
        mf              --  Matrix Factorization with an arbitrary loss
        BPRMF           --  Bayesian Personalized Ranking with MF
                            (just calls mf with logLoss)
        RankMFX         --  MF with hingeLoss
        slopeone        --  slopeone recommender
        svd             --  Ranking SVD (Sparse SVD)

PACKAGE CONTENTS
    BPRMF
    RankMFX
    knn
    mf
    nonpersonalized
    slopeone
    svd
\end{lstlisting}

\begin{lstlisting}
Help on module bin.recommender.knn in bin.recommender:

NAME
    bin.recommender.knn - Two classes for k-Nearest-Neighbor(knn) recommendation.

FILE
    /home/jk/recsyslab/bin/recommender/knn.py

DESCRIPTION
    itemKnn --  Item based knn
    userKnn --  User based knn
    
    Based on:
    George Karypis. 2001. Evaluation of Item-Based Top-N Recommendation
    Algorithms. In Proceedings of the tenth international conference on
    Information and knowledge management (CIKM 01), Henrique Paques,
    Ling Liu, and David Grossman (Eds.). ACM, New York, NY, USA, 247-254.
    DOI=10.1145/502585.502627 http://doi.acm.org/10.1145/502585.502627

CLASSES
    __builtin__.object
        itemKnn
        userKnn
    
    class itemKnn(__builtin__.object)
     |  Class for item based knn.
     |  
     |  Methods defined here:
     |  
     |  __init__(self, userItemMatrix, n)
     |      Builds a model for Item based knn.
     |      
     |          userItemMatrix  --  A matrix where an entry at i, j is the rating
     |                              the ith user gave the jth item.
     |          n               --  number of neighbors which are getting computed.
     |      
     |      Uses the cosine for similarity.
     |  
     |  getRec(self, u, n)
     |      Returns the n best recommendations for user u.
     |      
     |      Set n = -1 to get all items recommended
     |  
     |  ------------------------------------------------------------
     |  Data descriptors defined here:
     |  
     |  __dict__
     |      dictionary for instance variables (if defined)
     |  
     |  __weakref__
     |      list of weak references to the object (if defined)
    
    class userKnn(__builtin__.object)
     |  Methods defined here:
     |  
     |  __init__(self, userItemMatrix, n)
     |      Builds a model for user based knn.
     |      
     |          userItemMatrix  --  A matrix where an entry at i, j is the rating
     |                              the ith user gave the jth item.
     |          n               --  number of neighbors which get computed.
     |      
     |      Uses the cosine for similarity.
     |  
     |  getRec(self, u, n)
     |      Returns the n best recommendations for user u.
     |      
     |      Set n = -1 to get all items recommended
     |  
     |  ------------------------------------------------------------
     |  Data descriptors defined here:
     |  
     |  __dict__
     |      dictionary for instance variables (if defined)
     |  
     |  __weakref__
     |      list of weak references to the object (if defined)

FUNCTIONS
    computeCosSim(sim, matrix)
        Computes the cos of every two lines in the matrix and writes them into sim.
    
    cos(a, b)
        Gets two vectors(onedimensional np.matrix) and computes their cosine.
\end{lstlisting}
\newpage
\begin{lstlisting}
Help on module bin.recommender.mf in bin.recommender:

NAME
    bin.recommender.mf - Module for Matrix Factorization (mf) techniques.

FILE
    /home/jk/recsyslab/bin/recommender/mf.py

DESCRIPTION
    learnModel  --  learn a mf model
    MFrec       --  compute recommendations based on a mf model

CLASSES
    __builtin__.object
        MFrec
    
    class MFrec(__builtin__.object)
     |  Class to compute recommendations with a mf model.
     |  
     |  Methods defined here:
     |  
     |  __init__(self, W, H, trainingR)
     |      Initialize the class.
     |      
     |      W           --  Matrix of the User features
     |      H           --  Matrix of the Item features
     |      trainingR   --  The dict the model was trained with
     |  
     |  getRec(self, u, n)
     |      Returns the n best recommendations for user u based on the mf model.
     |      
     |      Set n = -1 to get all items recommended
     |  
     |  ------------------------------------------------------------
     |  Data descriptors defined here:
     |  
     |  __dict__
     |      dictionary for instance variables (if defined)
     |  
     |  __weakref__
     |      list of weak references to the object (if defined)

FUNCTIONS
    learnModel(n_users, m_items, regU, regI, regJ, learningRate, R, features, epochs, numberOfIterations, lossF, dlossF)
        Learns a mf model with a passed loss function.
        
            n_users             --  The highest internal assigned User ID
            m_items             --  The highest internal assigned Item ID
            regU                --  Regularization for the user vector
            regI                --  Regularization for the positive item
            regJ                --  Regularization for the negative item
            learningRate        --  The learning rate
            R                   --  A dict of the form UserID -> (ItemId, Rating)
            features            --  Number of features of the items and users
            epochs              --  Number of epochs the model should be learned
            numberOfIterations  --  Number of iterations in each epoch
            lossF               --  Loss function
            dlossF              --  Derivation of lossF
        
        Returns:
            W   --  User Features
            H   --  Item Features
        
        Based on:
        Steffen Rendle, Christoph Freudenthaler, Zeno Gantner, and
        Lars Schmidt-Thieme. 2009. BPR: Bayesian personalized ranking from
        implicit feedback. In Proceedings of the Twenty-Fifth Conference on
        Uncertainty in Artificial Intelligence (UAI 09).
        AUAI Press, Arlington, Virginia, United States, 452-461.
\end{lstlisting}

\begin{lstlisting}
Help on module bin.recommender.BPRMF in bin.recommender:

NAME
    bin.recommender.BPRMF

FILE
    /home/jk/recsyslab/bin/recommender/BPRMF.py

FUNCTIONS
    dLogLoss(a, y)
        Derivative of the logLoss.
    
    learnModel(n_users, m_items, regU, regI, regJ, learningRate, R, features, epochs, numberOfIterations)
        Learns a model with the BPRMF algorithm, actually just calls mf.
        
            n_users             --  The highest internal assigned User ID
            m_items             --  The highest internal assigned Item ID
            regU                --  Regularization for the user vector
            regI                --  Regularization for the positive item
            regJ                --  Regularization for the negative item
            learningRate        --  The learning rate
            R                   --  A dict of the form UserID -> (ItemId, Rating)
            features            --  Number of features of the items and users
            epochs              --  Number of epochs the model should be learned
            numberOfIterations  --  Number of iterations in each epoch
        
        Returns:
            W   --  User Features
            H   --  Item Features
        
        Based on:
        Steffen Rendle, Christoph Freudenthaler, Zeno Gantner, and
        Lars Schmidt-Thieme. 2009. BPR: Bayesian personalized ranking from
        implicit feedback. In Proceedings of the Twenty-Fifth Conference on
        Uncertainty in Artificial Intelligence (UAI 09).
        AUAI Press, Arlington, Virginia, United States, 452-461.
    
    logLoss(a, y)
        logLoss(a, y) = log(1 + exp(-a*y))
\end{lstlisting}

\begin{lstlisting}
Help on module bin.recommender.RankMFX in bin.recommender:

NAME
    bin.recommender.RankMFX

FILE
    /home/jk/recsyslab/bin/recommender/RankMFX.py

FUNCTIONS
    dHingeLoss(a, y)
        Derivative of the hingeLoss.
    
    hingeLoss(a, y)
        hingeLoss(a, y) = max(0, 1 - a*y)
    
    learnModel(n_users, m_items, regU, regI, regJ, learningRate, R, features, epochs, numberOfIterations)
        Learns a model with the RankMFX algorithm, actually just calls mf.
        
            n_users             --  The highest internal assigned User ID
            m_items             --  The highest internal assigned Item ID
            regU                --  Regularization for the user vector
            regI                --  Regularization for the positive item
            regJ                --  Regularization for the negative item
            learningRate        --  The learning rate
            R                   --  A dict of the form UserID -> (ItemId, Rating)
            features            --  Number of features of the items and users
            epochs              --  Number of epochs the model should be learned
            numberOfIterations  --  Number of iterations in each epoch
        
        Returns:
            W   --  User Features
            H   --  Item Features
\end{lstlisting}

\begin{lstlisting}
Help on module bin.recommender.slopeone in bin.recommender:

NAME
    bin.recommender.slopeone

FILE
    /home/jk/recsyslab/bin/recommender/slopeone.py

CLASSES
    __builtin__.object
        slopeone
    
    class slopeone(__builtin__.object)
     |  A class to compute recommendations with a Slope One Predictor.
     |  
     |  Based on:
     |      "Slope One Predictors for Online Rating-Based Collaborative Filtering"
     |      by Daniel Lemire and Anna Maclachlan.
     |  
     |  Methods defined here:
     |  
     |  __init__(self, R)
     |      Generate the model.
     |      
     |      R   --  A dict of the form UserID -> (ItemId, Rating)
     |  
     |  getRec(self, u, n)
     |      Returns the n best recommendations for user u.
     |      
     |      Set n = -1 to get all items recommended
     |  
     |  ------------------------------------------------------------
     |  Data descriptors defined here:
     |  
     |  __dict__
     |      dictionary for instance variables (if defined)
     |  
     |  __weakref__
     |      list of weak references to the object (if defined)
\end{lstlisting}

\begin{lstlisting}
Help on module bin.recommender.svd in bin.recommender:

NAME
    bin.recommender.svd - Module to compute the model of Ranking SVD (Sparse SVD)

FILE
    /home/jk/recsyslab/bin/recommender/svd.py

DESCRIPTION
    Based on:
    Jahrer, Michael, and Andreas Toescher.
    "Collaborative filtering ensemble for ranking."
    Proc. of KDD Cup Workshop at 17th ACM SIGKDD Int. Conf.
    on Knowledge Discovery and Data Mining, KDD. Vol. 11. 2011.

FUNCTIONS
    learnModel(n_users, m_items, learningRate, R, features, epochs, numberOfIterations)
        Returns the model of a learned svd as two matrices.
        
        n_users             --  The highest internal assigned User ID
        m_items             --  The highest internal assigned Item ID
        learningRate        --  The learning rate
        R                   --  A dict of the form UserID -> (ItemId, Rating)
        features            --  Number of features of the items and users
        epochs              --  Number of epochs the model should be learned
        numberOfIterations  --  Number of iterations in each epoch
\end{lstlisting}

\begin{lstlisting}
Help on package bin.util in bin:

NAME
    bin.util - Package with utility code to run the recommender algorithms.

FILE
    /home/jk/recsyslab/bin/util/__init__.py

DESCRIPTION
    It contains the following modules:
        reader  --  A reader to read in databases
        split   --  Split databases
        test    --  Test metrics
        helper  --  Several helping methods and a two way dict

PACKAGE CONTENTS
    helper
    reader
    split
    test
\end{lstlisting}

\begin{lstlisting}
Help on module bin.util.reader in bin.util:

NAME
    bin.util.reader - Contains a class to manage a database.

FILE
    /home/jk/recsyslab/bin/util/reader.py

CLASSES
    __builtin__.object
        stringSepReader
    
    class stringSepReader(__builtin__.object)
     |  A class to manage a database.
     |  
     |  Methods defined here:
     |  
     |  __init__(self, filename, separator)
     |      Reads in a database file.
     |      
     |      The lines of the database file have to look like this:
     |          UserID<separator>ItemID<separator>Rating
     |      If there is just an UserID and an ItemID the rating is set to 1.
     |      Everything coming after the rating will be ignored,
     |      but when omit the rating and still have something following the ItemID
     |      this will be understood as the rating.
     |  
     |  getInternalIid(self, originalIid)
     |      Maps the given original ItemID to the corresponding internal ItemID.
     |      
     |      The passed original UserID has to be a string.
     |  
     |  getInternalUid(self, originalUid)
     |      Maps the given original UserID to the corresponding internal UserID.
     |      
     |      The passed original UserID has to be a string.
     |  
     |  getMatrix(self)
     |      Get the database as a matrix.
     |      
     |      The lines of the matrix are corresponding to the users
     |      and the columns to the items.
     |      So the n,m entry is the rating the nth user gave the mth item.
     |  
     |  getMaxIid(self)
     |      Returns the highest internal assigned ItemID.
     |  
     |  getMaxUid(self)
     |      Returns the highest internal assigned UserID.
     |  
     |  getOriginalIid(self, internalIid)
     |      Maps the given internal ItemID to the corresponding original ItemID.
     |  
     |  getOriginalUid(self, internalUid)
     |      Maps the given internal UserID to the corresponding original UserID.
     |  
     |  getR(self)
     |      Return the database as a dict.
     |      
     |      The dict has internal UserIDs as keys and
     |      (ItemID, Rating) Tuples as values
     |  
     |  ------------------------------------------------------------
     |  Data descriptors defined here:
     |  
     |  __dict__
     |      dictionary for instance variables (if defined)
     |  
     |  __weakref__
     |      list of weak references to the object (if defined)
\end{lstlisting}

\begin{lstlisting}
Help on module bin.util.split in bin.util:

NAME
    bin.util.split - Methods to split a database with the leave-one-out method.

FILE
    /home/jk/recsyslab/bin/util/split.py

DESCRIPTION
    split        --  splits a reader into two new dicts
    splitMatrix  --  splits a matrix into a matrix and a dict

FUNCTIONS
    split(R, seed)
        Splits a database into two dicts.
        
        Splits after the leave-one-out method which means for every user in the
        database it takes one item out of the database and into a new one.
        The first returned dict is the database with one item missing for each
        user. The second returned dict has the missing items.
        
        R is a dict of the database.
        seed is a seed for the randomness.
    
    splitMatrix(M, seed)
        Splits a matrix into two new dicts.
        
        Returns an User x Item Matrix with one entry of each user set to 0
        and a User -> Item Dict with the missing entrys.
        
        M is an User x Item Matrix.
        seed is a seed for the randomness.
\end{lstlisting}

\begin{lstlisting}
Help on module bin.util.test in bin.util:

NAME
    bin.util.test - Module with several metrics to test the recommender algorithms.

FILE
    /home/jk/recsyslab/bin/util/test.py

DESCRIPTION
    hitrate     --  Returns #hits, #items
    precision   --  Returns #hits / n
    f1          --  Returns the result of a F1 test
    mrhr        --  Returns the Mean Reciprocal Hitrate
    auc         --  Returns the Area under the curve
    countHits   --  Returns hits, items in a tuple

FUNCTIONS
    auc(testR, recommender, r)
        Returns and prints the AUC of the recommender function.
        
        testR is a dict with an internal UserID as a dict and a list of items as
        values. Normally testR is the second dict split.split returns.
        The list can have a lenght greater than 1.
        
        recommender is a function which takes an internal UserID and n and returns
        n items recommender for the UserID.
        
        r is a reader object with the database read in.
        
        Based on:
        Steffen Rendle, Christoph Freudenthaler, Zeno Gantner, and
        Lars Schmidt-Thieme. 2009. BPR: Bayesian personalized ranking from
        implicit feedback. In Proceedings of the Twenty-Fifth Conference on
        Uncertainty in Artificial Intelligence (UAI 09).
        AUAI Press, Arlington, Virginia, United States, 452-461.
    
    countHits(testR, recommender, n)
        Returns the number of hits and the number of items it was tested with.
        
        Is a helper function for the other metrics
        testR is a dict with an internal UserID as a dict and a list of items as
        values. Normally testR is the second dict split.split returns.
        The list can have a lenght greater than 1.
        
        recommender is a function which takes an internal UserID and n and returns
        n items recommender for the UserID.
        
        n is the number of items the recommender can recommend.
        
        hits = number of items from testR the recommender guessed right.
        items = the number of items in testR.
        
        hits, items will be printed.
        hits, items will get returned.
    
    f1(testR, recommender, n)
        Prints and returns F1 of the recommender.
        
        F1 = (2 * hitrate * precision) / (hitrate + precision)
        testR is a dict with an internal UserID as a dict and a list of items as
        values. Normally testR is the second dict split.split returns.
        The list can have a lenght greater than 1.
        
        recommender is a function which takes an internal UserID and n and returns
        n items recommender for the UserID.
        
        n is the number of items the recommender can recommend.
        
        hits = number of items from testR the recommender guessed right.
        
        See also: "Application of Dimensionality Reduction in Recommender System
        -- A Case Study" by Badrul M. sarcar et al.
    
    hitrate(testR, recommender, n)
        Returns the number of hits and the number of items it was tested with.
        
        testR is a dict with an internal UserID as a dict and a list of items as
        values. Normally testR is the second dict split.split returns.
        The list can have a lenght greater than 1.
        
        recommender is a function which takes an internal UserID and n and returns
        n items recommender for the UserID.
        
        n is the number of items the recommender can recommend.
        
        hits = number of items from testR the recommender guessed right.
        items = the number of items in testR.
        
        hits, items and hits / items will be printed.
        hits, items will get returned.
        
        hits / items gives the hitrate or recall.
        
        See also:
        Sarwar, Badrul, et al. Application of dimensionality reduction in
        recommender system-a case study. No. TR-00-043.
        MINNESOTA UNIV MINNEAPOLIS DEPT OF COMPUTER SCIENCE, 2000.
    
    mrhr(testR, recommender, n)
        Returns the Mean Reciprocal Hitrate of the recommender.
        
        testR is a dict with an internal UserID as a dict and a list of items as
        values. Normally testR is the second dict split.split returns.
        The list can have a length greater than 1.
        
        recommender is a function which takes an internal UserID and n and returns
        n items recommender for the UserID.
        
        n is the number of items the recommender can recommend.
        
        hits = number of items from testR the recommender guessed right.
    
    precision(testR, recommender, n)
        Returns the number of hits / n .
        
        testR is a dict with an internal UserID as a dict and a list of items as
        values. Normally testR is the second dict split.split returns.
        The list can have a lenght greater than 1.
        
        recommender is a function which takes an internal UserID and n and returns
        n items recommender for the UserID.
        
        n is the number of items the recommender can recommend.
        
        hits = number of items from testR the recommender guessed right.
        
        See also:
        Sarwar, Badrul, et al. Application of dimensionality reduction in
        recommender system-a case study. No. TR-00-043.
        MINNESOTA UNIV MINNEAPOLIS DEPT OF COMPUTER SCIENCE, 2000.
\end{lstlisting}

\begin{lstlisting}
Help on module bin.util.helper in bin.util:

NAME
    bin.util.helper - Several helper functions and a helper class.

FILE
    /home/jk/recsyslab/bin/util/helper.py

DESCRIPTION
    idOrigDict          --  two way dict
    normRowNorm         --  normalize a matrix by the norm
    normRowSum          --  normalize a matrix by the sum
    cache               --  cache results of a function
    writeOrigToFile     --  save a DB with original IDs
    writeInternalToFile --  save a DB with internal IDs
    sortList            --  sort a list with scores and IDs
    listToSet           --  convert a list into a set
    sortResults         --  sort test results
    printMatrix         --  print a matrix
    randDataset         --  generate a random DB
    dictToMatrix        --  convert a dict to a matrix
    getScoreMF          --  compute the score of an item
    getExternalRec      --  convert getRec for original IDs

CLASSES
    __builtin__.object
        idOrigDict
    
    class idOrigDict(__builtin__.object)
     |  Class two map external/original IDs to internal IDs and vice versa.
     |  
     |  Methods defined here:
     |  
     |  __init__(self)
     |      Initialization.
     |  
     |  add(self, orig)
     |      Add a new original ID.
     |      
     |      Returns the internal ID the passed original ID got mapped to.
     |      If the passed ID is already mapped nothing happens except the
     |      mapped internal ID is returned.
     |  
     |  getId(self, orig)
     |      Returns the internal ID which is mapped to the passed orinigal ID.
     |  
     |  getOrig(self, id)
     |      Returns the original ID which is mapped to the passed internal ID.
     |  
     |  ----------------------------------------------------------------------
     |  Data descriptors defined here:
     |  
     |  __dict__
     |      dictionary for instance variables (if defined)
     |  
     |  __weakref__
     |      list of weak references to the object (if defined)

FUNCTIONS
    cache(fn)
        Decorator to cache the result of the function fn.
    
    dictToMatrix(d)
        Converts a dict like reader.getR() returns to a matrix.
    
    getExternalRec(getRec, r)
        Converts getRec so it takes and returns only the original IDs.
        
        r   --  reader object
    
    getScoreMF(origUid, origIid, W, H, r)
        Returns the score of one item for one user with an MF model.
        origUid --  original User ID
        origIid --  original Item ID
        W, H    --  The MF model like returned by recommender.mf for example
        r       --  a reader object with the database
    
    listToSet(sortedscorelist, n)
        Converts a list into a set with size n.
        
        Gets a list like sortList returns and returns a set with the first n
        items, or less, when there are not enough items
    
    normRowNorm(m)
        Normalize the matrix in place so each row of the matrix has norm 1.
    
    normRowSum(m)
        Normalize the matrix in place so each row of the matrix has sum 1.
    
    printMatrix(m)
        Prints a whole matrix.
        
        Prints the whole matrix m without replacing stuff with dots
        like it normally does when the matrix is to large.
    
    randDataset(user, items, p, seed, filename)
        Generates a random dataset and writes it into a file.
        
        user    --  number of users
        items   --  number of items
        p       --  percentage of ones in the dataset
        seed    --  seed for the randomness
    
    sortList(scorelist)
        Gets a list of tuples (itemid, score), sorts by score decreasing.
    
    sortResults(name)
        Sort test results written to a file.
    
    writeInternalToFile(reader, toSave, filename)
        Writes a database into a file with the internal IDs.
        
            reader  --  reader object
            toSave  --  Part of the database to write
            filename--  Name of the file
        
        Writes toSave into a file with the name filename.
    
    writeOrigToFile(reader, toSave, filename)
        Writes a database into a file with the original IDs.
        
            reader  --  reader object
            toSave  --  Part of the database to write
            filename--  Name of the file
        
        Writes toSave into a file with the name filename but first the
        internal IDs in toSave get mapped to the original IDs.
\end{lstlisting}
