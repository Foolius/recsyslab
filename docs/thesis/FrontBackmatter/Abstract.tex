%*******************************************************
% Abstract
%*******************************************************
%\renewcommand{\abstractname}{Abstract}
\pdfbookmark[1]{Abstract}{Abstract}
\begingroup
\let\clearpage\relax
\let\cleardoublepage\relax
\let\cleardoublepage\relax

\chapter*{Abstract}
In the context of this thesis, we implemented a recommender system library called \textit{recsyslab}~\cite{recsyslab}. We used the programming language Python~\cite{python} to this end, given its simplicity and elegance. The goal of recsyslab is
to provide easy access to current recommendation algorithms, by being easy to use and 
having easily readable source code.
The development of recsyslab is the major contribution of this thesis. 
This document is intended to document its design and facilitate its use.
%The user guide in Chapter \ref{usermanual} shows that we have been able to make the library easy to use
%and the test results in Chapter \ref{experiments} show that it also performs well.


\vfill

\pdfbookmark[1]{Zusammenfassung}{Zusammenfassung}
\chapter*{Zusammenfassung}
Im Rahmen dieser Bachelorarbeit haben wir eine Bibliothek namens \textit{recsyslab}~\cite{recsyslab} 
in der Programmiersprache Python~\cite{python} implementiert. Wir haben Python aufgrund dessen Einfachheit und Eleganz benutzt. Das Ziel von recsyslab ist es, einen m\"oglichst einfachen Zugang zu 
Empfehlungsalgorithmen zu gew\"ahren, indem es einfach zu benutzen ist und der Quellcode gut zu lesen ist.
Die Entwicklung von recsyslab ist der Hauptteil der Bachelorarbeit.
Dieses Dokument soll ihre Struktur dokumentieren und ihren Gebrauch vereinfachen.
%Die Bedienungsanleitung in Kapitel \ref{usermanual} zeigt, dass es uns gelungen ist die Bibliothek einfach
%bedienbar zu gestalten und die Testergebnisse in Kapitel \ref{experiments} zeigen, dass sie gut funktioniert.


\endgroup			

\vfill

\cleardoublepage
